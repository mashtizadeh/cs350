%% -*- Lecture -*-

\documentclass[11pt,aspectratio=169]{beamer}

\usepackage{rcstalk}
\usetheme{rcstheme}

\topic{Advanced File Systems}
\subtitle{Lecture 13: Advanced File Systems}

\usepackage{tikz}
\usetikzlibrary{chains,shadows.blur}

\tikzset{
    only/.code args={<#1>#2}{\only<#1>{\pgfkeysalso{#2}}},
}

\newcommand\tikznode[3][anchor=base]%
{\tikz[remember picture,baseline,inner sep=0]{\node[#1] (#2) {#3};}}
\newcommand\annotate[4][(label) -- (text)]%
{\begin{tikzpicture}[inner sep=0,baseline=(text.base)]
\node (text) {#2};
\node[red,overlay] (label) at (#3) {\sffamily #4};
\draw[->,thick,red,overlay] #1;
\end{tikzpicture}}

\begin{document}

\maketitle

\section{FFS in more detail}

\begin{slide}{Review: FFS background}
\itms{
  \item 1980s improvement to original Unix FS, which had:
  \ittms{
    \item 512-byte blocks
    \item Free blocks in linked list
    \item All inodes at beginning of disk
    \item Low throughput:  512 bytes per average seek time
  }
  \item Unix FS performance problems:
  \ittms{
    \item Transfers only 512 bytes per disk access
    \item Eventually random allocation $\to$ 512 bytes / disk seek
    \item Inodes far from directory and file data
    \item Within directory, inodes far from each other
  }
  \item Also had some usability problems:
  \ittms{
    \item 14-character file names a pain
    \item Can't atomically update file in crash-proof way
  }
}
\end{slide}


\begin{slide}{Review: FFS
    \cref{readings/ffs.pdf}{[McKusic]}
        basics}
\itms{
  \item Change block size to at least 4K
  \ittms{
    \item To avoid wasting space, use ``fragments'' for ends of files
  }
  \item Cylinder groups spread inodes around disk
  \item Bitmaps replace free list
  \item FS reserves space to improve allocation
  \ittms{
    \item Tunable parameter, default 10\%
    \item Only superuser can use space when over 90\% full
  }
  \item Usability improvements:
  \ittms{
    \item File names up to 255 characters
    \item Atomic \emph{rename} system call
    \item Symbolic links assign one file name to another
  }
}
\end{slide}

\begin{slide}{Review: FFS disk layout}
\centerline{\input{ffs.tex}}
\vspace*{.1in}
\itms{
  \item Each cylinder group has its own:
  \ittms{
    \item Superblock
    \item Bookkeeping information
    \item Set of inodes
    \item Data/directory blocks
  }
}
\end{slide}

\begin{slide}{Superblock}
\itms{
  \item Contains file system parameters
  \ittms{
    \item Disk characteristics, block size, CG info
    \item \Red{Information necessary to locate inode given i-number}
  }
  \item Replicated once per cylinder group
  \ittms{
    \item At shifting offsets, so as to span multiple platters
    \item Contains magic number \texttt{0x011954} to find replicas if
      1st superblock dies (Kirk McKusick's birthday?)
  }
  \item Contains non-replicated ``summary information''
  \ittms{
    \item \# blocks, fragments, inodes, directories in FS
    \item Flag stating if FS was cleanly unmounted
  }
}
\end{slide}

\begin{frame}
\frametitle{Bookkeeping information}
\begin{itemize}
  \item Block map
  \begin{itemize}
    \item Bit map of available fragments
    \item Used for allocating new blocks/fragments
  \end{itemize}
  \item Summary info within CG
  \begin{itemize}
    \item \# free inodes, blocks/frags, files, directories 
    \item Used when picking cylinder group from which to allocate
  \end{itemize}
  \item \# free blocks by rotational position (8 positions)
  \begin{itemize}
    \item Was reasonable in 1980s when disks weren't commonly zoned
    \item Back then OS could do stuff to minimize rotational delay
  \end{itemize}
\end{itemize}
\end{frame}

\begin{slide}{Inodes and data blocks}
\centerline{\input{inode.tex}}
\vspace*{-3mm}
\itms{
  \item Each CG has fixed \# of inodes (default one per 2K data)
  \item Each inode maps \alert{offset $\to$ disk block} for one file
  \item An inode also contains metadata for its file
  \begin{itemize}
    \item permissions, access/modification/change times, link
      count, \ldots
  \end{itemize}
%  \item By convention, inode \#2 always root directory
}
\end{slide}

%% \begin{slide}{Cylinder groups}
%% \itms{
%%   \item Groups related inodes and their data
%%   \item Contains fixed number of inodes (set when FS created)
%%   \ittms{
%%     \item Default one inode per 2K data
%%   }
%%   \item Contains file and directory blocks
%%   \item Contains bookkeeping information
%% }
%% \end{slide}

\begin{slide}{Inode allocation}
\itms{
  \item Each file or directory created requires a new inode
  \item New file?  Put inode in same CG as directory if possible
  \item New directory?  Use different CG from parent
  \ittms{
    \item Consider CGs with greater than average \# free inodes
    \item Chose CG with smallest \# directories
  }
  \item Within CG, inodes allocated randomly (next free)
  \ittms{
    \item Would like related inodes as close as possible
    \item OK, because one CG doesn't have that many inodes
    \item All inodes in CG can be read and cached with small \# of reads
  }
}
\end{slide}

\begin{slide}{Fragment allocation}
\itms{
  \item Allocate space when user writes beyond end of file
  \item Want last block to be a fragment if not full-size
  \ittms{
    \item If already a fragment, may contain space for write -- done
    \item Else, must deallocate any existing fragment, allocate new
  }
  \item If no appropriate free fragments, break full block
  \item Problem:  Slow for many small writes
  \begin{itemize}
    \item May have to keep moving end of file around
  \end{itemize}
  \item (Partial) soution:  new \texttt{stat} struct field \texttt{st\_blksize}
  \ittms{
    \item Tells applications file system block size
    \item stdio library can buffer this much data
  }
}
\end{slide}

\begin{slide}{Block allocation}
\itms{
  \item Try to optimize for sequential access
  \ittms{
    \item If available, use rotationally close block in same
	    cylinder (obsolete)
    \item Otherwise, use block in same CG
    \item If CG totally full, find other CG with quadratic hashing \\
      i.e., if CG \#$n$ is full, try $n+1^2, n+2^2,
      n+3^2,\ldots \pmod{\# CGs}$
    \item Otherwise, search all CGs for some free space
  }
  \item Problem:  Don't want one file filling up whole CG
  \ittms{
    \item Otherwise other inodes will have data far away
  }
  \item Solution:  Break big files over many CGs
  \ittms{
    \item But large extents in each CGs, so sequential access
	doesn't require many seeks
    \item How big should extents be?
\pause
    \item Extent transfer time should be much greater than seek time
  }
}
\end{slide}

\begin{slide}{Directories}
\itms{
  \item Inodes like files, but with different type bits
  \item Contents considered as 512-byte \textit{chunks}
  \item Each chunk has \texttt{direct} structure(s) with:
  \ittms{
    \item 32-bit inumber
    \item 16-bit size of directory entry
    \item 8-bit file type (added later)
    \item 8-bit length of file name
  }
  \item Coalesce when deleting
  \ittms{
    \item If first \texttt{direct} in chunk deleted, set inumber = 0
  }
  \item Periodically compact directory chunks
  \begin{itemize}
    \item But can never move directory entries across chunks
    \item Recall only 512-byte sector writes atomic w.\ power failure
  \end{itemize}
}
\end{slide}

\begin{slide}{Updating FFS for the 90s}
\itms{
  \item No longer wanted to assume rotational delay
  \ittms{
    \item With disk caches, want data contiguously allocated
  }
  \item Solution:  Cluster writes
  \ittms{
    \item FS delays writing a block back to get more blocks
    \item Accumulates blocks into 64K clusters, written at once
  }
  \item Allocation of clusters similar to fragments/blocks
  \ittms{
    \item Summary info
    \item Cluster map has one bit for each 64K if all free
  }
  \item Also read in 64K chunks when doing read ahead
}
\end{slide}

\section{Crash recoverability}

\begin{slide}{Fixing corruption -- fsck}
\itms{
  \item Must run FS check (fsck) program after crash
  \item Summary info usually bad after crash
  \ittms{
    \item Scan to check free block map, block/inode counts
  }
  \item System may have corrupt inodes (not simple crash)
  \ittms{
    \item Bad block numbers, cross-allocation, etc.
    \item Do sanity check, clear inodes with garbage
  }
  \item Fields in inodes may be wrong
  \ittms{
    \item Count number of directory entries to verify link count,
	  if no entries but count $\ne$ 0, move to \texttt{lost+found}
    \item Make sure size and used data counts match blocks
  }
  \item Directories may be bad
  \ittms{
    \item Holes illegal, \texttt{.} and \texttt{..} must be valid,
      file names must be unique
    \item All directories must be reachable
  }
}
\end{slide}

\begin{slide}{Crash recovery permeates FS code}
\itms{
  \item \Red{Have to ensure fsck can recover file system}
  \item Example: Suppose all data written asynchronously
  \begin{itemize}
    \item Any subset of data structures may be updated before a crash
  \end{itemize}
  \item Delete/truncate a file, append to other file, crash
  \ittms{
    \item New file may reuse block from old
    \item Old inode may not be updated
    \item Cross-allocation!
    \item Often inode with older mtime wrong, but can't be sure
  }
  \item Append to file, allocate indirect block, crash
  \ittms{
    \item Inode points to indirect block
    \item But indirect block may contain garbage!
  }
}
\end{slide}

\begin{slide}{Ordering of updates}
\itms{
  \item Must be careful about order of updates
  \ittms{
    \item Write new inode to disk before directory entry
    \item Remove directory name before deallocating inode
    \item Write cleared inode to disk before updating CG free map
%    \item Don't reuse on-disk struct before nullifying pointers to it
%   \item Don't create persistent pointers to uninitialized structs
  }
  \item Solution:  Many metadata updates synchronous
  \ittms{
    \item Doing one write at a time ensures ordering
    \item Of course, this hurts performance
    \item E.g., untar much slower than disk bandwidth
  }
  \item Note:  \Red{Cannot update buffers on the disk queue}
  \ittms{
    \item E.g., say you make two updates to same directory block
    \item But crash recovery requires first to be synchronous
    \item Must wait for first write to complete before doing second
  }
}
\end{slide}

\begin{slide}{Performance vs.\ consistency}
\itms{
  \item FFS crash recoverability comes at \emph{huge} cost
  \ittms{
    \item Makes tasks such as untar easily 10-20 times slower
    \item All because you \emph{might} lose power or reboot at any time
  }
  \item Even while slowing ordinary usage, recovery slow
  \ittms{
    \item If fsck takes one minute, then disks get 10$\times$ bigger \ldots
  }
  \item One solution:  battery-backed RAM
  \ittms{
    \item Expensive (requires specialized hardware)
    \item Often don't learn battery has died until too late
    \item A pain if computer dies (can't just move disk)
    \item If OS bug causes crash, RAM might be garbage
  }
  \item Better solution:  Advanced file system techniques
  \ittms{
    \item Topic of rest of lecture
  }
}
\end{slide}

\section{Soft updates}

\begin{slide}{First attempt:  Ordered updates}
\itms{
  \item Want to avoid crashing after ``bad'' subset of writes \\
  \item Must follow 3 rules in ordering updates
    \cref{readings/softupdates.pdf}{[Ganger]}:
  \ittms{
    \item[1.] Never write pointer before initializing the structure it
    points to 
    \item[2.] Never reuse a resource before nullifying all pointers to it
    \item[3.] Never clear last pointer to live resource before setting new one
  }
  \item If you do this, file system will be recoverable
  \item Moreover, can recover quickly
  \ittms{
    \item Might leak free disk space, but otherwise correct
    \item So start running after reboot, scavenge for space in background
  }
  \item How to achieve?
  \ittms{
    \item Keep a partial order on buffered blocks
  }
}
\end{slide}

\begin{slide}{Ordered updates (continued)}
\itms{ 
  \item Example:  Create file $A$
  \ittms{
    \item Block $X$ contains an inode
    \item Block $Y$ contains a directory block
    \item Create file $A$ in inode block $X$, dir block $Y$
  }
  \item We say \Red{$Y\to X$}, pronounced ``$Y$ \emph{depends on} $X$''
  \ittms{
    \item Means $Y$ cannot be written before $X$ is written
    \item $X$ is called the \Red{depend\emph{ee}}, $Y$ the
      \Red{depend\emph{er}}
  }
  \item Can delay both writes, so long as order preserved
  \ittms{
    \item Say you create a second file $B$ in blocks $X$ and $Y$
    \item Only have to write each out once for both creates
  }
}
\end{slide}

\begin{slide}{Problem:  Cyclic dependencies}
\itms{
  \item Suppose you create file $A$, unlink file $B$
  \ittms{
    \item Both files in same directory block \& inode block
  }
  \item Can't write directory until $A$'s inode initialized
  \ittms{
    \item Otherwise, after crash directory will point to bogus inode
    \item Worse yet, same inode \# might be re-allocated
    \item So could end up with file name $A$ being an unrelated file
  }
  \item Can't write inode block until $B$'s directory entry cleared
  \ittms{
    \item Otherwise, $B$ could end up with too small a link count
    \item File could be deleted while links to it still exist
  }
  \item Otherwise, fsck has to be slow
  \ittms{
    \item Check every directory entry and inode link count
  }
}
\end{slide}

\tikzset{
    box/.style={draw, minimum width=2cm, text height=1.75ex, text
        depth=.25ex,fill=white},
    free/.style={fill=green!30},
    dirty/.style={text=red,font=\sffamily\bfseries},
    topdownboxes/.style={start chain=going below, node distance=0pt,
                   every node/.append style={on chain,box,outer sep=0pt}}
}

\begin{frame}
\frametitle{Cyclic dependencies illustrated}
\begin{tikzpicture}[font=\normalfont\normalsize\sffamily]
\path[topdownboxes]
   node[draw=none] {inode block}
   node[free] {inode \#4}
   node       {inode \#5}
   node[free] {inode \#6}
   node       {inode \#7};
\path[topdownboxes] (3,0)
   node[draw=none] {directory block}
   node[free] {$\langle$--,\#0$\rangle$}
   node       {$\langle$B,\#5$\rangle$}
   node       {$\langle$C,\#7$\rangle$};
\node at (1.5,-3.25) {Original organization};

\path[rounded corners,fill=yellow!20,thick,draw=yellow,blur shadow]
(5,0) rectangle (10.5,-2.25);

\path[topdownboxes, node distance=2mm]
   (6.5,-.75) node {in use} node[free] {free};
\path[topdownboxes, node distance=2mm]
   (9, -.75) node {original} node[dirty] {modified};
\draw[very thick,dashed] (7.75,-.25) -- (7.75,-2);

\tikzset{yshift=-4cm}

\path[topdownboxes] (0,0)
   node[draw=none] {inode block}
   node[dirty,name=i4b] {inode \#4}
   node       {inode \#5}
   node[free] {inode \#6}
   node       {inode \#7};
\path[topdownboxes] (3,0)
   node[draw=none] {directory block}
   node[dirty] {$\langle$A,\#4$\rangle$} edge[very thick,->] (i4b)
   node       {$\langle$B,\#5$\rangle$}
   node       {$\langle$C,\#7$\rangle$};
\node at (1.5,-3.25) {Create file A};

\path[topdownboxes] (6,0)
   node[draw=none] {inode block}
   node[dirty,name=i4c] {inode \#4}
   node[dirty,free,name=i5c] {inode \#5}
   node[free] {inode \#6}
   node       {inode \#7};
\path[topdownboxes] (9,0)
   node[draw=none] {directory block}
   node[dirty] {$\langle$A,\#4$\rangle$} edge[very thick,->] (i4c)
   node[dirty,free] {$\langle$--,\#5$\rangle$} edge[very thick,<-] (i5c)
   node       {$\langle$C,\#7$\rangle$};
\node at (7.5,-3.25) {Remove file B};
\end{tikzpicture}
\end{frame}

\begin{slide}{More problems}
\itms{
  \item Crash might occur between ordered but related writes
  \ittms{
    \item E.g., summary information wrong after block freed
  }
  \item Block aging
  \ittms{
    \item Block that always has dependency will never get written back
  }
  \item Solution:  \emph{Soft updates}
    \cref{readings/softupdates.pdf}{[Ganger]}
  \ittms{
    \item Write blocks in any order
    \item But keep track of dependencies
    \item When writing a block, temporarily roll back any changes 
	  you can't yet commit to disk
    \item I.e., can't write block with any arrows pointing to
      dependees \\
      \ldots but can temporarily undo whatever change requires the
      arrow
  }
}
\end{slide}

\begin{frame}<1-4>
\frametitle{Breaking dependencies with rollback}
\begin{tikzpicture}[font=\normalfont\normalsize\sffamily]
\node[font=\bfseries\Large] at (1.5,.75) {Buffer cache};
\path[topdownboxes] (0,0)
   node[draw=none] {inode block}
   node[dirty,name=i4c] {inode \#4}
   node[dirty,free,name=i5c] {inode \#5}
   node[free] {inode \#6}
   node       {inode \#7};
\path[topdownboxes] (3,0)
   node[draw=none] {directory block}
   node[dirty] (d4c) {$\langle$A,\#4$\rangle$}
   node[dirty,free] (d5c) {$\langle$--,\#0$\rangle$}
   node       {$\langle$C,\#7$\rangle$};
\only<1-2>{\draw[very thick,->] (d4c) -- (i4c);}
\only<1>{\draw[very thick,<-] (d5c) -- (i5c);}
\tikzset{xshift=6cm}
\node[font=\bfseries\Large] at (1.5,.75) {Disk};
\path[topdownboxes]
   node[draw=none] {inode block}
   node[only=<-2>{free},only=<3->{dirty}] {inode \#4}
   node (diski5)      {inode \#5}
   node[free] {inode \#6}
   node       {inode \#7};
\path[topdownboxes] (3,0)
   node[draw=none] {directory block}
   node[free] (diska) {$\langle$--,\#0$\rangle$}
   node (diskb) {$\langle$B,\#5$\rangle$}
   node       {$\langle$C,\#7$\rangle$};
\only<2->{ \node[box,free,dirty] at (diskb.center) {$\langle$--,\#0$\rangle$}; }
\only<3->{ \node[box,free,dirty] at (diski5.center) {inode \#5}; }
\only<4->{ \node[box,dirty] at (diska.center) {$\langle$A,\#4$\rangle$}; }
\end{tikzpicture}
\bigskip
\begin{overprint}
\onslide<1>
\itms{
  \item Deleted Created file A and deleted file B
  \item Now say we decide to write directory block\ldots
  \item Can't write file name $A$ to disk---has dependee
}
\onslide<2>
\itms{
  \item Undo file $A$ before writing dir block to disk
  \ittms{
    \item Even though we just wrote it, directory block still dirty
%      \rnode{todirty}{dirty}
%\ncdiag[linecolor=blue,linewidth=.08,armA=1,linearc=.3,angleB=315,armB=.7,nodesep=.1]{->}{todirty}{dirty}
  }
  \item But now inode block has no dependees
  \ittms{
    \item Can safely write inode block to disk as-is\ldots
  }
}
\onslide<3>
\itms{
  \item Now inode block clean (same in memory as on disk)
  \item But have to write directory block a second time\ldots
}
\onslide<4>
\itms{
  \item All data stably on disk
  \item Crash at any point would have been safe
}
\end{overprint}
\end{frame}

%% \begin{frame}
%% \frametitle{Breaking dependencies w.\ rollback}
%% \begin{overprint}
%% \onslide<1>\centerline{\input{su1}}
%% \itms{
%%   \item Now say we decide to write directory block\ldots
%%   \item Can't write file name $A$ to disk---has dependee
%% }
%% \onslide<2>\centerline{\input{su2}}
%% \itms{
%%   \item Undo file $A$ before writing dir block to disk
%%   \ittms{
%%     \item Even though we just wrote it, directory block still dirty
%% %      \rnode{todirty}{dirty}
%% %\ncdiag[linecolor=blue,linewidth=.08,armA=1,linearc=.3,angleB=315,armB=.7,nodesep=.1]{->}{todirty}{dirty}
%%   }
%%   \item But now inode block has no dependees
%%   \ittms{
%%     \item Can safely write inode block to disk as-is\ldots
%%   }
%% }
%% \onslide<3>\centerline{\input{su3}}
%% \itms{
%%   \item Now inode block clean (same in memory as on disk)
%%   \item But have to write directory block a second time\ldots
%% }
%% \onslide<4>\centerline{\input{su4}}
%% \itms{
%%   \item All data stably on disk
%%   \item Crash at any point would have been safe
%% }
%% \end{overprint}
%% \end{frame}

\begin{slide}{Soft updates}
\itms{
  \item Structure for each updated field or pointer, contains:
  \ittms{
    \item old value
    \item new value
    \item list of updates on which this update depends (\emph{dependees})
  }
  \item Can write blocks in any order
  \ittms{
    \item But must temporarily undo updates with pending dependencies
    \item Must lock rolled-back version so applications don't see it
    \item Choose ordering based on disk arm scheduling
  }
  \item Some dependencies better handled by postponing in-memory updates
  \ittms{
    \item E.g., when freeing block (e.g., because file truncated),
      just mark block free in bitmap after block pointer cleared on disk
  }
}
\end{slide}

\begin{slide}{\hypertarget{simple-example}{Simple example}}
\itms{
  \item Say you create a zero-length file $A$
  \item Depender:  Directory entry for $A$
  \ittms{
    \item Can't be written untill dependees on disk
  }
  \item Dependees:
  \ittms{
    \item Inode -- must be initialized before dir entry written
    \item Bitmap -- must mark inode allocated before dir entry written
  }
  \item Old value:  empty directory entry
  \item New value:  $\langle \mathrm{filename}\ A, \mathrm{inode}\
    \#\rangle$
  \item Can write directory block to disk any time
  \ittms{
    \item Must substitute old value until inode \& bitmap updated on disk
    \item Once dir block on disk contains $A$, file fully created
    \item Crash before $A$ on disk, worst case might leak the inode
  }
}
\end{slide}

\begin{slide}{Operations requiring soft updates (1)}
\itms{
  \item[1.] Block allocation
  \ittms{
       \item Must write the disk block, the free map, \& a pointer
       \item Disk block \& free map must be written before pointer
       \item Use Undo/redo on pointer (\& possibly file size)
  }
  \item[2.] Block deallocation
  \ittms{
       \item Must write the cleared pointer \& free map
       \item Just update free map after pointer written to disk
       \item Or just immediately update free map if pointer not on disk 
  }
  \item Say you quickly append block to file then truncate
  \ittms{
    \item You will know pointer to block not written because of the
	  allocated dependency structure
    \item So both operations together require no disk I/O!
  }
}
\end{slide}

\begin{slide}{Operations requiring soft updates (2)}
\itms{
  \item[3.] Link addition (see \hyperlink{simple-example}{simple
      example})
  \ittms{
       \item Must write the directory entry, inode, \& free map (if new inode)
       \item Inode and free map must be written before dir entry
       \item Use undo/redo on i\# in dir entry (ignore entries w.\ i\# 0)
  }
  \item[4.] Link removal
  \ittms{
       \item Must write directory entry, inode \& free map (if nlinks==0)
       \item Must decrement nlinks only after pointer cleared
       \item Clear directory entry immediately
       \item Decrement in-memory nlinks once pointer written
       \item If directory entry was never written, decrement
  immediately \\
         (again will know by presence of dependency structure)
  }
  \item Note:  Quick create/delete requires no disk I/O
}
\end{slide}

\begin{slide}{Soft update issues}
\itms{
  \item \emph{fsync} -- sycall to flush file changes to disk
  \ittms{
    \item Must also flush directory entries, parent directories, etc.
  }
  \item \emph{unmount} -- flush all changes to disk on shutdown
  \ittms{
    \item Some buffers must be flushed multiple times to get clean
  }
  \item Deleting large directory trees frighteningly fast
  \ittms{
    \item \emph{unlink} syscall returns even if inode/indir block
    not cached!
    \item Dependencies allocated faster than blocks written
    \item Cap \# dependencies allocated to avoid exhausting memory
  }
  \item Useless write-backs
  \ittms{
    \item Syncer flushes dirty buffers to disk every 30 seconds
    \item Writing all at once means many dependencies unsatisfied
    \item Fix syncer to write blocks one at a time
    \item Fix LRU buffer eviction to know about dependencies
  }
}
\end{slide}

\begin{slide}{Soft updates fsck}
\itms{
 \item Split into foreground and background parts
 \item Foreground must be done before remounting FS
 \ittms{
        \item Need to make sure per-cylinder summary info makes sense
	\item Recompute free block/inode counts from bitmaps -- very fast
	\item Will leave FS consistent, but might leak disk space
  }
  \item Background does traditional fsck operations
  \ittms{
      \item  Do after mounting to recuperate free space
      \item  Can be using the file system while this is happening
      \item  Must be done in forground after a media failure
  }
  \item Difference from traditional FFS fsck:
  \ittms{
           \item May have many, many inodes with non-zero link counts
           \item Don't stick them all in lost+found (unless media failure)
  }
}
\end{slide}

\section{Journaling}

\begin{slide}{An alternative: Journaling}
%\vspace*{-.15in}
\itms{
  \item Biggest crash-recovery challenge is inconsistency
  \ittms{
    \item Have one logical operation (e.g., create or delete file)
    \item Requires multiple separate disk writes
    \item If only some of them happen, end up with big problems
  }
  \item Most of these problematic writes are to metadata
  \item Idea:  Use a \emph{write-ahead} log to \emph{journal}
    metadata
  \ittms{
    \item Reserve a portion of disk for a log
    \item Write any metadata operation first to log, then to disk
    \item After crash/reboot, re-play the log (efficient)
    \item May re-do already committed change, but won't miss anything
  }
} 
\end{slide}

\begin{slide}{Journaling (continued)}
\itms{
  \item Group multiple operations into one log entry
  \ittms{
    \item E.g., clear directory entry, clear inode, update free
      map---\\ either all three will happen after recovery, or none
  }
  \item Performance advantage:
  \ittms{
    \item Log is consecutive portion of disk
    \item Multiple operations can be logged at disk b/w
    \item Safe to consider updates committed when written to log
  }
  \item Example:  delete directory tree
  \ittms{
    \item Record all freed blocks, changed directory entries in log
    \item Return control to user
    \item Write out changed directories, bitmaps, etc.\ in background
      \\ (sort for good disk arm scheduling)
  }
}
\end{slide}

\begin{slide}{Journaling details}
\itms{
  \item Must find oldest relevant log entry
  \ittms{
    \item Otherwise, redundant and slow to replay whole log
  }
  \item Use checkpoints
  \ittms{
    \item Once all records up to log entry $N$ have been processed and
	  affected blocks stably committed to disk\ldots
    \item Record $N$ to disk either in reserved checkpoint location, or
    in checkpoint log record
    \item Never need to go back before most recent checkpointed $N$
  }
  \item Must also find end of log
  \ittms{
    \item Typically circular buffer; don't play old records out of order
    \item Can include begin transaction/end transaction records
    \item Also typically have checksum in case some sectors bad
  }
}
\end{slide}

\begin{slide}{Case study: XFS
    \cref{readings/xfs.pdf}{[Sweeney]}}
\itms{
  \item Main idea:  Think big
  \ittms{
    \item Big disks, files, large \# of files, 64-bit everything
    \item Yet maintain very good performance
  }
  \item Break disk up into \emph{Allocation Groups} (AGs)
  \ittms{
    \item 0.5 -- 4 GB regions of disk
    \item New directories go in new AGs
    \item Within directory, inodes of files go in same AG
    \item Unlike cylinder groups, AGs too large to minimize seek times 
    \item Unlike cylinder groups, no fixed \# of inodes per AG
  }
  \item Advantages of AGs:
  \ittms{
    \item Parallelize allocation of blocks/inodes on multiprocessor \\
(independent locking of different free space structures)
    \item Can use 32-bit block pointers within AGs \\
(keeps data structures smaller)
  }
}
\end{slide}

\begin{slide}{B+-trees}   
\centerline{\input{btree.tex}}
\itms{
  \item XFS makes extensive use of B+-trees
  \ittms{
    \item Indexed data structure stores ordered Keys \& Values
    \item Keys must have an ordering defined on them
    \item Stored data in blocks for efficient disk access
  }
  \item For B+-tree with $n$ items, all operations $O(\log n)$:
  \ittms{
    \item Retrieve closest $\langle$key, value$\rangle$ to target key $k$
    \item Insert a new $\langle$key, value$\rangle$ pair
    \item Delete $\langle$key, value$\rangle$ pair
  }
}
\end{slide}

\begin{slide}{B+-trees continued}
\itms{
  \item See any algorithms book for details (e.g., [Cormen])
  \item Some operations on B-tree are complex:
  \ittms{
    \item E.g., insert item into completely full B+-tree
    \item May require ``splitting'' nodes, adding new level to tree
    \item Would be bad to crash \& leave B+tree in inconsistent state
  }
  \item Journal enables atomic complex operations
  \ittms{
    \item First write all changes to the log
    \item If crash while writing log, incomplete log record will be
	  discarded, and no change made
    \item Otherwise, if crash while updating B+-tree, will replay
	  entire log record and write everything
  }
}
\end{slide}

\begin{slide}{B+-trees in XFS}
\itms{
  \item B+-trees are complex to implement
  \ittms{
    \item But once you've done it, might as well use everywhere
  }
  \item Use B+-trees for directories (keyed on filename hash)
  \ittms{
    \item Makes large directories efficient
  }
  \item Use B+-trees for inodes
  \ittms{
    \item No more FFS-style fixed block pointers
    \item Instead, B+-tree maps:  file offset $\to$ $\langle$start
	  block, \# blocks$\rangle$ 
    \item Ideally file is one or small number of contiguous extents
    \item Allows small inodes \& no indirect blocks even for huge files
  }
  \item Use to find inode based on inumber
  \ittms{
    \item High bits of inumber specify AG
    \item B+-tree in AG maps:  starting i\# $\to$ $\langle$block \#,
      free-map$\rangle$
    \item So free inodes tracked right in leaf of B+-tree
  }
}
\end{slide}

\begin{slide}{More B+-trees in XFS}
\itms{
  \item Free extents tracked by \emph{two} B+-trees
  \ittms{
    \item[1.] start block \# $\to$ \# free blocks
    \item[2.] \# free blocks $\to$ start block \#
  }
  \item Use journal to update both atomically \& consistently
  \item \#1 allows you to coalesce adjacent free regions
  \item \#1 allows you to allocate near some target
  \ittms{
    \item E.g., when extending file, put next block near previous one
    \item When first writing to file, but data near inode
  }
  \item \#2 allows you to do best fit allocation
  \ittms{
    \item Leave large free extents for large files
  }
}
\end{slide}

\begin{slide}{Contiguous allocation}
\itms{
  \item Ideally want each file contiguous on disk
  \ittms{
    \item Sequential file I/O should be as fast as sequential disk I/O
  }
  \item But how do you know how large a file will be?
  \item Idea:  \Red{delayed allocation}
  \ittms{
    \item \emph{write} syscall only affects the buffer cache
    \item Allow write into buffers before deciding where to place on disk
    \item Assign disk space only when buffers are flushed
  }
  \item Other advantages:
  \ittms{
    \item Short-lived files never need disk space allocated
    \item \emph{mmap}ed files often written in random order in memory,
  but will be written to disk mostly contiguously 
    \item Write clustering:  find other nearby stuff to write to disk
  }
}
\end{slide}

\begin{slide}{Journaling vs.\ soft updates}
\itms{
  \item Both much better than FFS alone
  \item Some limitations of soft updates
  \ittms{
    \item Very specific to FFS data structures (E.g., couldn't easily
      add B-trees like XFS---even directory rename not quite right)
    \item Metadata updates may proceed out of order (E.g., create $A$,
      create $B$, crash---maybe only $B$ exists after reboot)
    \item Still need slow background fsck to reclaim space
  }
  \item Some limitations of journaling
  \ittms{
    \item Disk write required for every metadata operation (whereas
      create-then-delete might require no I/O with soft updates)
    \item Possible contention for end of log on multi-processor
    \item \emph{fsync} must sync other operations' metadata to log, too
  }
}
\end{slide}

\end{document}
